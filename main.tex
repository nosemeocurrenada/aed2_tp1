\documentclass[10pt,a4paper]{article}

\input{src/AEDmacros}
\usepackage{src/caratula} % Version modificada para usar las macros de algo1 de ~> https://github.com/bcardiff/dc-tex
\newcommand{\letramate}[1]{\ensuremath{\normalfont\bfseries\ttfamily #1}}

\titulo{Trabajo práctico 1: Especificación y WP}
\subtitulo{Subtítulo del tp}

\fecha{\today}

\materia{Algoritmos y Estructuras de Dato}
\grupo{Grupo 5}

\integrante{Perez Vargas, Pedro Ignacio}{672/21}{pedroignacioperezvargas@gmail.com}
\integrante{Romero Ávila, Luis}{826/18}{luisr.romeroa@gmail.com}
\integrante{Meza Alarcón, Cristian}{835/15}{cmeza@dc.uba.ar}
\integrante{Julian Sesto}{1948/21}{sestojulian@gmail.com}

\graphicspath{{./static/}}

\begin{document}

\maketitle

\section{Especificación}

\begin{enumerate} \setlength\itemsep{0cm}
	\item 
        \begin{proc}{redistribucionDeLosFrutos}{
            \In recursos : \TLista{\float}, 
            \In cooperan : \TLista{\bool}
        }{\TLista{\float}}
        	%    \modifica{parametro1, parametro2,..}
            \requiere{
                |recursos| = |cooperan|
            }
            \asegura{
                |res| = |recursos|
                \yLuego (\forall i : \nat)\left(0 \leq i < |res| \implicaLuego res[i] = \dfrac{sumaRecursos(recursos, cooperan)}{|recursos|}\right)
            }
            \aux{sumaRecursos}{recursos, cooperan}{\float}{
                \sum_{i=0}^{|recursos|-1} \IfThenElse{cooperan[i]}{recursos[i]}{0}
            }
        \end{proc}
	\item 
        \begin{proc}{trayectoriaDeLosFrutosIndividualesALargoPlazo}{
            \Inout trayectorias : \TLista{\TLista{\float}}, 
            \In    cooperan : \TLista{\bool}, 
            \In    apuestas : \TLista{\TLista{\float}},
            \In    pagos : \TLista{\TLista{\float}},
            \In    eventos : \TLista{\TLista{\nat}}
        }{}
        	%    \modifica{parametro1, parametro2,..}
            \requiere{
                |trayectorias| = |cooperan| = |eventos| = |apuestas| = |pagos| \\
                %% Pido esto para poder usar |eventos[0]| = #pasos
                \land |eventos| \geq 1 \\
                \land (\forall i : \nat)(0 \leq i < |trayectorias| \implicaLuego |trayectorias[i]| = 1) \\
                \land (\forall i : \nat)(0 \leq i < |apuestas| \implicaLuego apuestaValida(apuestas[i]) \yLuego |apuestas[i]| = |pagos[i]|) \\
                \land (\forall i, t : \nat)(0 \leq i < |eventos| \implicaLuego
                    0 \leq t < |eventos[i]| \implicaLuego 0 \leq eventos[i][t] < |apuestas[i]|
                ) \\
                \land (\forall i, j : \nat)(0 \leq i, j < |eventos| \implicaLuego |eventos[i]| = |eventos[j]|) \\
            }
            \asegura{
                    \texttt{trayectoriaLargoPlazo}(trayectorias, cooperan, apuestas, pagos, eventos, |trayectorias|, |eventos[0]|)
            }
         	\aux{gananciaIndividual}{
                    ev : \nat,
                    apuestas: \TLista{\float},
                    pagos : \TLista{\float},
                    rec : \float
                    }
                    {\float}
                    {apuestas[ev] * pagos[ev] * rec}
                \aux{fondo}{
                    trayectorias : \TLista{\TLista{\float}}, 
                    cooperan : \TLista{\bool}, 
                    apuestas : \TLista{\TLista{\float}},
                    pagos : \TLista{\TLista{\float}}, \\
                    eventos : \TLista{\TLista{\nat}},
                    \#individuos : \nat,
                    t : \nat
                    }
                    {\float}
                    {\\
                    \displaystyle \sum^{\#individuos-1}_{i=0} \IfThenElse{cooperan[i]}
                        {
                        \texttt{gananciaIndividual}(eventos[i][t-1], apuestas[i], pagos[i],\\
                        trayectorias[i][t-1])
                        }
                        {
                            0
                        }
                    }
            %\aux{\#individuos}{}{\nat}{|trayectorias|}
            %\aux{\#pasos}{}{\nat}{|eventos[0]|}
        	\pred{apuestaValida}
                {apuesta : \TLista{\float}}
                {\sum_{i=0}^{|apuesta|-1} apuesta[i] = 1 \land (\forall i : \nat)(0 \leq i < |apuesta| \implicaLuego apuesta[i] \geq 0 )
                } 
            \pred{trayectoriaLargoPlazo}
                {
                trayectorias : \TLista{\TLista{\float}}, 
                cooperan : \TLista{\bool}, 
                apuestas : \TLista{\TLista{\float}},\\
                pagos : \TLista{\TLista{\float}},
                eventos : \TLista{\TLista{\nat}},
                \#individuos : \nat,
                \#pasos : \nat
                }
                {
                    (\forall i : \nat)(0 \leq i < \#individuos
                    \implicaLuego (\forall t : \nat)(1 \leq t < \#pasos
                    \implicaLuego \\
                    |trayectorias[i]| = 1 + \#pasos
                    \yLuego \\
                    trayectorias[i][t] = \dfrac{fondo(trayectorias, cooperan, apuestas, pagos, eventos, \#individuos, t)}{\#individuos}\\
                    + \IfThenElse{cooperan[i]}{
                        0 \\
                    }{
                        \texttt{gananciaIndividual}(eventos[i][t-1], apuestas[i], pagos[i], trayectorias[i][t-1])
                        \\
                    }
                    ))
                }
        \end{proc}
	\item 
        \begin{proc}{trayectoriaExtrañaEscalera}{
            \In trayectoria : \TLista{\float}
        }{\bool}
         	\requiere{|trayectoria| > 1}
         	\asegura{res = \True \leftrightarrow \texttt{existeMaximoLocal}(trayectoria)}
         	% \aux{auxiliar1}{parametros}{tipoRes}{expresion}
        	\pred{existeMaximoLocal}{s : \TLista{\float}}{(\exists i: \ent)
          \left(1 \leq i < |s|-1
          \yLuego 
          (s[i] > s[i-1] \land s[i] > s[i+1])
          \lor (s[0] > s[1]) 
          \lor (s[|s|-1] > s[|s|-2])
          \right)} 
        \end{proc}

    \item 
        \begin{proc}{individuoDecideSiCooperarONo}{
            \In individuo : \nat,
            \In recursos : \TLista{\float}, 
            \Inout cooperan : \TLista{\bool}, \\
            \In apuestas : \TLista{\TLista{\float}},
            \In pagos : \TLista{\TLista{\float}},
            \In eventos : \TLista{\TLista{\nat}}
        }{}
        \requiere{
            |recursos| = |cooperan| = |eventos| = |apuestas| = |pagos| \\
            %% 0 \leq i < |rec| es falso siempre que |rec| = 0
            %% entonces esto implica |eventos| = |rec| > 1
            %% y por lo tanto vale pedir eventos[0]
            \land 0 \leq individuo < |recursos|\\
            \land (\forall i : \nat)(0 \leq i < |apuestas| \implicaLuego apuestaValida(apuestas[i]) \yLuego |apuestas[i]| = |pagos[i]|) \\
            \land (\forall i, t : \nat)(0 \leq i < |eventos| \implicaLuego
                0 \leq t < |eventos[i]| \implicaLuego 0 \leq eventos[i][t] < |apuestas[i]|
            ) \\
            \land (\forall i, j : \nat)(0 \leq i, j < |eventos| \implicaLuego |eventos[i]| = |eventos[j]|\\
            \land cooperan = cooperan_0 \\
        }
        \asegura{
            cooperan = \IfThenElse{
                \texttt{esMejorCooperar}(individuo, recursos, cooperan_0, apuestas, pagos, eventos, \\
                |recursos|, |eventos[0]|)
                \\
            }{
                \texttt{setAt}(cooperan_0, individuo, \True)
                \\
            }{
                \texttt{setAt}(cooperan_0, individuo, \False)
                \\
            } \\
        }

        \pred{esMejorCooperar}{
            individuo : \nat,
            recursosIniciales : \TLista{\float}, 
            $\text{cooperan}_0$ : \TLista{\bool}, 
            apuestas : \TLista{\TLista{\float}},
            pagos : \TLista{\TLista{\float}},
            eventos : \TLista{\TLista{\nat}},
            \#individuos : \nat,
            \#pasos : \nat
        }{
            (\exists tc, tn : \TLista{\TLista{\float}})(
            \\
            \texttt{coincidenRecursosIniciales}(tc, recursosIniciales) 
            \\
            \land \texttt{coincidenRecursosIniciales}(tn, recursosIniciales) 
            \\
            \land \texttt{trayectoriaLargoPlazo}(
                tc,
                \texttt{setAt}(cooperan_0, individuo, \True),
                apuestas,
                pagos,
                eventos,
                \#individuos,
                \#pasos
            ) 
            \\
            \land \texttt{trayectoriaLargoPlazo}(
                tn,
                \texttt{setAt}(cooperan_0, individuo, \False),
                apuestas,
                pagos,
                eventos,
                \#individuos,
                \#pasos
            ) 
            \\
            \land tc[individuo][\#pasos] \geq tn[individuo][\#pasos])
        }
            
        
        \pred{coincidenRecursosIniciales}{
            trayectorias : \TLista{\TLista{\float}},
            recursosIniciales : \TLista{\float}
        }{
            (\forall i : \nat)(0 \leq i < |recursosIniciales| \implicaLuego
            trayectorias[i][0] = recursosIniciales[i])
        }
    \end{proc}

    \item \begin{proc}{individuoActualizaApuesta}{
            \In individuo : \nat,
            \In recursos : \TLista{\float}, 
            \In cooperan : \TLista{\bool}, 
            \Inout apuestas : \TLista{\TLista{\float}},
            \In pagos : \TLista{\TLista{\float}},
            \In eventos : \TLista{\TLista{\nat}}
        }{}
        	%    \modifica{parametro1, parametro2,..}
            \requiere{
                |recursos| = |cooperan| = |eventos| = |apuestas| = |pagos| \\
                \land 0 \leq individuo < |recursos| \\
                \land (\forall i : \nat)(0 \leq i < |apuestas| \implicaLuego apuestaValida(apuestas[i]) \yLuego |apuestas[i]| = |pagos[i]|) \\
                \land (\forall i, t : \nat)(0 \leq i < |eventos| \implicaLuego
                    0 \leq t < |eventos[i]| \implicaLuego 0 \leq eventos[i][t] < |apuestas[i]|
                ) \\
                \land (\forall i, j : \nat)(0 \leq i, j < |eventos| \implicaLuego |eventos[i]| = |eventos[j]|\\
                \land apuestas = apuestas_0 \\
            }
         	\asegura{
                \texttt{noModificaOtrasApuestas}(apuestas_0,apuestas,individuo) \\
                \land (\exists t_M : \TLista{\TLista{\float}})(
                \texttt{todasApuestasValidas}(apuestas) \\
                \land \texttt{coincidenRecursosIniciales}(t_M, recursos) \\
                \land \texttt{trayectoriaLargoPlazo}(
                    t_M,
                    cooperan,
                    apuestas,
                    pagos,
                    eventos
                ) \\
                \yLuego \texttt{esTrayectoriaMaximal}(
                    t_M,
                    individuo,
                    cooperan,
                    pagos,
                    eventos,
                    |recursos|,
                    |eventos[0]|
                ) \\
            }
                \pred{esTrayectoriaMaximal}{
                    $\text{t}_M$ : \TLista{\TLista{\float}}
                    individuo : \nat,
                    cooperan : \TLista{\bool}, 
                    pagos : \TLista{\TLista{\float}},
                    eventos : \TLista{\TLista{\nat}},
                    \#individuos : \nat,
                    \#pasos : \nat
                }{
                    (\forall t,ap: \TLista{\TLista{\float}})(
                    \texttt{todasApuestasValidas}(ap) \\
                    \land \texttt{trayectoriaLargoPlazo}(
                        t,
                        cooperan,
                        ap,
                        pagos,
                        eventos,
                        \#individuos,
                        \#pasos
                    ) \\
                    \implicaLuego
                    (t_M[individuo][\#pasos] \geq t[individuo][\#pasos])) \\
                }
                
                \pred{todasApuestasValidas}{
                    apuestas: \TLista{\TLista{\float}}
                }{
                    (\forall i : \nat)(0 \leq i < |apuestas| \implicaLuego apuestaValida(apuestas[i]))
                }

                \pred{noModificaOtrasApuestas}{
                    $\text{apuestas}_0$ : \TLista{\TLista{\float}},
                    apuestas : \TLista{\TLista{\float}},
                    individuo : \nat,
                }{
                    |apuestas| = |apuestas_0| \\
                    \yLuego
                    (\forall i : \nat)(i \neq individuo \land 0 \leq i < |apuestas| \implicaLuego secuenciasCoinciden(apuestas[i],apuestas_0[i])) 
                }
                
                \pred{secuenciasCoinciden}{
                    $\text{s}_1$: \TLista{\float},
                    $\text{s}_2$: \TLista{\float}
                }{
                    |s_1| = |s_2|
                    \yLuego
                    (\forall i : \nat)(0 \leq i < |s_1|
                    \implicaLuego
                        s_1[i] = s_2[i]
                    )
                }
            
         	% \aux{auxiliar1}{parametros}{tipoRes}{expresion}
        	% \pred{pred1}{parametros}{expresion} 
        \end{proc}

        \end{enumerate}

\section{Demostraciones de correctitud}
\begin{proc}{frutoDelTrabajoPuramenteIndividual}{
    \In recurso : \float,
    \In apuesta : $\langle$ s : \float, c : \float $\rangle$, 
    \In pago : $\langle$ s : \float, c : \float $\rangle$,
    \In evento : \TLista{\bool},
    \Out res : \float
}{}
    %    \modifica{parametro1, parametro2,..}
    \requiere{expresionBooleana1}
    \asegura{expresionBooleana2}
    \aux{auxiliar1}{parametros}{tipoRes}{expresion}
    \pred{pred1}{parametros}{expresion} 
\end{proc}
% \begin{proc}{nombre}{\In paramIn : \nat, \Inout paramInout : \TLista{\ent}}{tipoRes}
	%    \modifica{parametro1, parametro2,..}
% 	\requiere{expresionBooleana1}
% 	\asegura{expresionBooleana2}
% 	\aux{auxiliar1}{parametros}{tipoRes}{expresion}
% 	\pred{pred1}{parametros}{expresion} 
% \end{proc}

% \aux{auxiliarSuelto}{parametros}{tipoRes}{expresion}
% \paraTodo{variable}{tipo}{expresion}
% \existe{variable}{tipo}{expresion}
% Pueden tener [unalinea] para que no se divida en varias lineas
% \pred{predSuelto}{parametros}{\paraTodo[unalinea]{variable}{tipo}{algo \implicaLuego expresion}}
% \pred{predSuelto}{parametros}{\existe[unalinea]{variable}{tipo}{algo \yLuego expresion}}

\end{document}
